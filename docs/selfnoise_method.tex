\documentclass[12pt]{amsart}
\usepackage{amsmath,amsfonts,amsthm, amscd, mathrsfs, amssymb}


\usepackage{fullpage}

\linespread{1.6}


\begin{document}


\title{Method for Self-Noise Estimates in Broadband Seismometers}
\author{Adam T. Ringler}
\author{Rafael Figueroa-Briceno}
\date{\today}
\address{U.S. Geological Survey, Albuquerque Seismological Laboratory, P.O. Box 82010, Albuquerque, NM 87198-2010}
\email{aringler@usgs.gov} 

\maketitle




\section{Introduction}

\subsection{Self-Noise Test Setup}
Depending on the type of instrument and frequencies of interest a number of different test configurations
and methods have been developed.  Our goal here is to simply give the setup guidelines for routine broadband 
self-noise estimates.  We do not focus on the more complex problem of better constraining the long-period self-noise
of a seismometer.

\begin{enumerate}

    \item Install three seismometers (2 references and 1 test sensor) in a a quiet vault.  The sensors should be installed as close as
    possible without interfering with additional thermal isolation.  This should include a fleece cap as well as an exterior foam box.
    % Include a Figure of an example installation
    
    \item Once the sensors have been installed being recording data an a digitizer with sufficient resolution to record the noise of the seismometers.
    To verify that your digitizer has sufficient resolution you can perform this self-noise test on data from the digitizer that has been obtained by 
    terminating the output with a 10 $\Omega$ resistor.  No part of the self-noise of the digitizer should be above the self-noise of the seismometer that you 
    will be testing.
    
    \item Record all sensors at 40 samples per second.  
    
    \item After the sensors have settled sufficientely collect 6 hours of data where there are no transients larger than XX above the standard deviation
    of the time window being used.
    
    \item Estimate the self-noise by computing the corss-powers between all three sensors in test.  For estimating the cross-power use a Welch method
    with a $5\%$ cosine taper, windows with $2^14$ points and $2^12$ points of overlap.  
    
    % Need to decide on the Welch method
    \item Using the nomainl response correct the power and the noise estimated in the previous step.  Your results should now be in units of $(m/s^2)^2 /Hz$.
    
    \item Convert from linear units to $dB$.
    
    \item Plot all results relative to the Peterson NLNM.
    % Include pass/fail criteria as well as statistics
    
\end{enumerate}





\begin{thebibliography}{10}
\providecommand{\url}[1]{{#1}}
\providecommand{\urlprefix}{URL }
\expandafter\ifx\csname urlstyle\endcsname\relax
  \providecommand{\doi}[1]{DOI~\discretionary{}{}{}#1}\else
  \providecommand{\doi}{DOI~\discretionary{}{}{}\begingroup
  \urlstyle{rm}\Url}\fi

% Add in all the additional references

\bibitem{Sleeman}
Sleeman, R., van Wettum, A., and Trampert, J.
\newblock Bull. Seism. Soc. Am. \textbf{96}, 258--271 (2006)


\end{thebibliography}




%\bibliographystyle{spmpsci}
%\bibliography{ref2.bib}   % name your BibTeX data base
%\nocite{*}
 


\end{document}
